 \documentclass[11pt]{article}

% Language setting
\usepackage[turkish]{babel}
\usepackage{pythonhighlight}

\usepackage[a4paper,top=2cm,bottom=2cm,left=2cm,right=2cm,marginparwidth=2cm]{geometry}

% Useful packages
\usepackage{amsmath}
\usepackage{graphicx}
\usepackage[colorlinks=true, allcolors=blue]{hyperref}
\usepackage{verbatim}
\usepackage{fancyhdr} % for header and footer
\usepackage{titlesec}
\usepackage{parskip}

\setlength{\parindent}{0pt}

\titleformat{\subsection}[runin]{\bfseries}{\thesubsection}{1em}{}

\pagestyle{fancy} % activate the custom header/footer

% define the header/footer contents
\lhead{\small{23BLM-4014 Yapay Sinir Ağları Ara Sınav Soru ve Cevap Kağıdı}}
\rhead{\small{Dr. Ulya Bayram}}
\lfoot{}
\rfoot{}

% remove header/footer on first page
\fancypagestyle{firstpage}{
  \lhead{}
  \rhead{}
  \lfoot{}
  \rfoot{\thepage}
}
 

\title{Çanakkale Onsekiz Mart Üniversitesi, Mühendislik Fakültesi, Bilgisayar Mühendisliği Akademik Dönem 2022-2023\\
Ders: BLM-4014 Yapay Sinir Ağları/Bahar Dönemi\\ 
ARA SINAV SORU VE CEVAP KAĞIDI\\
Dersi Veren Öğretim Elemanı: Dr. Öğretim Üyesi Ulya Bayram}
\author{%
\begin{minipage}{\textwidth}
\raggedright
Öğrenci Adı Soyadı: Mutlu Bora
Öğrenci No: 180401005
\end{minipage}%
}

\date{14 Nisan 2023}

\begin{document}
\maketitle

\vspace{-.5in}
\section*{Açıklamalar:}
\begin{itemize}
    \item Vizeyi çözüp, üzerinde aynı sorular, sizin cevaplar ve sonuçlar olan versiyonunu bu formatta PDF olarak, Teams üzerinden açtığım assignment kısmına yüklemeniz gerekiyor. Bu bahsi geçen PDF'i oluşturmak için LaTeX kullandıysanız, tex dosyasının da yer aldığı Github linkini de ödevin en başına (aşağı url olarak) eklerseniz bonus 5 Puan! (Tavsiye: Overleaf)
    \item Çözümlerde ya da çözümlerin kontrolünü yapmada internetten faydalanmak, ChatGPT gibi servisleri kullanmak serbest. Fakat, herkesin çözümü kendi emeğinden oluşmak zorunda. Çözümlerinizi, cevaplarınızı aşağıda belirttiğim tarih ve saate kadar kimseyle paylaşmayınız. 
    \item Kopyayı önlemek için Github repository'lerinizin hiçbirini \textbf{14 Nisan 2023, saat 15:00'a kadar halka açık (public) yapmayınız!} (Assignment son yükleme saati 13:00 ama internet bağlantısı sorunları olabilir diye en fazla ekstra 2 saat daha vaktiniz var. \textbf{Fakat 13:00 - 15:00 arası yüklemelerden -5 puan!}
    \item Ek puan almak için sağlayacağınız tüm Github repository'lerini \textbf{en geç 15 Nisan 2023 15:00'da halka açık (public) yapmış olun linklerden puan alabilmek için!}
    \item \textbf{14 Nisan 2023, saat 15:00'dan sonra gönderilen vizeler değerlendirilmeye alınmayacak, vize notu olarak 0 (sıfır) verilecektir!} Son anda internet bağlantısı gibi sebeplerden sıfır almayı önlemek için assignment kısmından ara ara çözümlerinizi yükleyebilirsiniz yedekleme için. Verilen son tarih/saatte (14 Nisan 2023, saat 15:00) sistemdeki en son yüklü PDF geçerli olacak.
    \item Çözümlerin ve kodların size ait ve özgün olup olmadığını kontrol eden bir algoritma kullanılacaktır. Kopya çektiği belirlenen vizeler otomatikman 0 (sıfır) alacaktır. Bu nedenle çözümlerinizi ve kodlarınızı yukarıda sağladığım gün ve saatlere kadar kimseyle paylaşmayınız.
    \item Bu vizeden alınabilecek en yüksek not 100'dür. Toplam aldığınız puan 100'ü geçerse, aldığınız not 100'e sabitlenecektir.
    \item LaTeX kullanarak PDF oluşturanlar öz geçmişlerine LaTeX bildiklerini de eklemeyi unutmasınlar :)
    \item Bu vizedeki soruların çözümleri ve tex dosyası için istediğiniz kadar sayıda Github repository'si oluşturabilirsiniz. Sadece yukarıda belirttiğim tarihlerde (14 Nisan 2023 saat 15:00 ile 15 Nisan 2023 saat 15:00 arasında) public yapmayı/halka açmayı ve Github profilinizi de öz geçmişinize eklemeyi unutmayın :)
    \item Bilgisayarınıza hiçbir program kurmadan, Overleaf, Google Colab, Kaggle gibi ortamlar üzerinden bu vizeyi tamamlamanız mümkün. İyi çalışmalar!!
\end{itemize}

(Ekstra 5 puan) \url{https://github.com/mutlubora/YapaySinirAglari}

\newpage
\section{(Toplam 10 Puan) Hopfield Networks:}

\subsection{(2 Puan)} \textbf{ChatGPT’ye size Hopfield Network’ler ile ilgili detaylı bilgi vermesini isteyin. Sorduğunuz soruyu, ChatGPT'nin çıkartığı cevabı aşağı koyun. Bu cevapta anlamadığınız terimleri de ChatGPT’ye sorup, sorduğunuz soruları ve ChatGPT’nin verdiği cevapları da sırayla aşağı koyun.}

\textbf{Hopfield Networkler hakkında detaylı bilgi ver.}

Hopfield ağları (Hopfield Networks), John Hopfield tarafından 1982 yılında tanımlanmış bir yapay sinir ağı türüdür. Hopfield ağları, özellikle hatırlama (recall) ve optimizasyon problemleri için kullanılan basit ve yaygın olarak kullanılan bir modeldir. Hopfield ağları, tam bağlantılı (fully connected) ve geri beslemeli (recurrent) bir ağdır.

Hopfield ağları, her bir nöronun iki durumunu (aktif ve pasif) alabilen ve simetrik olarak birbirleriyle bağlı olan nöronlardan oluşur. Nöronlar, birbirleriyle etkileşimde bulunarak, bir enerji fonksiyonu (energy function) minimize edilene kadar dinamik bir şekilde güncellenirler. Enerji fonksiyonu, ağdaki nöronların durumlarının bir işlevidir ve nöronların durumlarını optimize etmek için kullanılır.

Hopfield ağları, bir dizi örnek durumun (pattern) öğrenilmesi ve bu örneklerin sonradan hatırlanması için kullanılabilir. Bir Hopfield ağına örnek durumlar öğretildiğinde, ağ öğrendiği örnek durumları hatırlayabilir ve verilen bir girdiye göre en yakın örnek durumu hatırlayabilir.

Hopfield ağları ayrıca optimizasyon problemleri için de kullanılabilir. Bir enerji fonksiyonunun minimize edilmesi, optimizasyon problemlerinde sık sık karşılaşılan bir problemdir. Hopfield ağı, enerji fonksiyonunu optimize etmek için kullanılabilir ve ağdaki nöronların durumlarının minimize edilen enerji fonksiyonunu temsil ettiği bir çözüm sağlar.

Hopfield ağları, basit bir yapısı ve kolay uygulanabilirliği nedeniyle yaygın olarak kullanılır. Ancak, büyük ölçekli problemler için yeterli performansı sağlayamazlar ve bu nedenle daha büyük ve karmaşık ağlar kullanmak gerekebilir.

 \textbf{Burada bahsettiğin optimizasyon problemleri hakkında bilgi ve örnek ver.}

 Optimizasyon problemleri, bir hedef fonksiyonunun (objective function) maksimum ya da minimum değerinin bulunmasını gerektiren matematiksel problemlerdir. Bu tür problemler, pek çok alanda karşımıza çıkabilir ve genellikle optimizasyon algoritmaları kullanılarak çözülürler.

Örnek olarak, bir mühendislik problemi ele alalım. Bir yapı tasarımı yapılırken, yapıda kullanılacak malzemenin miktarı minimize edilmeye çalışılırken belirli bir mukavemet gereksinimini karşılaması istenebilir. Bu durumda, yapıyı tanımlayan bir matematiksel fonksiyon oluşturulur ve bu fonksiyon, kullanılan malzemenin miktarını minimize etmek için optimize edilir. Bu, bir optimizasyon probleminin bir örneğidir.

Başka bir örnek olarak, bir üretim sürecinde işletme maliyetlerinin minimize edilmesi hedeflenebilir. Bu durumda, üretim sürecini tanımlayan bir matematiksel model oluşturulur ve bu model üzerinde optimizasyon yapılır.

Optimizasyon problemlerinin çözülmesinde, Hopfield ağları gibi yapay sinir ağı modelleri kullanılabilir. Hopfield ağları, bir enerji fonksiyonunun minimize edilmesi için kullanılabilir ve bu nedenle optimizasyon problemlerini çözmek için kullanılabilir. Ancak, büyük ölçekli problemler için daha etkili optimizasyon algoritmaları kullanmak gerekebilir.

\textbf{"Hopfield ağları, tam bağlantılı (fully connected) ve geri beslemeli (recurrent) bir ağdır." cümlesindeki tam bağlantılı olmak ve geri beslemeli olmak ne anlama geliyor?}

Tam bağlantılı (fully connected) bir Hopfield ağı, her nöronun ağdaki tüm diğer nöronlarla bağlı olduğu anlamına gelir. Yani, her nöron her diğer nöronla bağlantılıdır. Bu nedenle, ağın toplam bağlantı sayısı, nöron sayısının karesi kadardır. Bu bağlantılar, nöronların durumlarını etkilemek için kullanılır.

Geri beslemeli (recurrent) bir Hopfield ağı, ağdaki nöronların birbirleriyle geri besleme yapabileceği anlamına gelir. Yani, bir nöronun çıkışı diğer nöronların girdisi olabilir ve bu nedenle nöronlar arasında geri besleme döngüleri oluşabilir. Bu geri besleme döngüleri, ağın dinamik davranışını etkiler ve özellikle hatırlama (recall) işlemi sırasında önemlidir.

Tam bağlantılı ve geri beslemeli bir Hopfield ağı, nöronların etkileşimlerinin tüm ağ boyunca yayıldığı bir yapıya sahiptir. Bu nedenle, ağın davranışı bir bütün olarak incelenir ve bir nöronun durumu, diğer nöronların durumuna bağlı olarak değişebilir. Bu, ağın hatırlama işleminde özellikle önemlidir, çünkü bir nöronun durumu, ağdaki diğer nöronların durumlarına bağlı olarak hatırlama işlemi sırasında güncellenir.

 \textbf{Bir enerji fonksiyonunun minimize edilmesi ne anlama geliyor?}

 Bir enerji fonksiyonunun minimize edilmesi, fonksiyonun en düşük (minimum) değerine ulaşmak için uygun parametrelerin bulunmasını ifade eder. Enerji fonksiyonları, genellikle bir sistemin enerjisiyle ilgili matematiksel modellerdir ve sistemin belirli bir durumu için enerjiyi hesaplarlar. Bu fonksiyonlar, yapay sinir ağlarından fiziksel sistemlere kadar pek çok farklı alanda kullanılabilir.

Örneğin, Hopfield ağlarında, enerji fonksiyonu ağın durumunu tanımlar ve ağın hedef durumuna ulaşmak için enerjinin minimize edilmesi gerekir. Bu nedenle, enerji fonksiyonu, ağın durumlarının optimize edilmesinde kullanılır.

Enerji fonksiyonlarının minimize edilmesi, belirli bir hedefe ulaşmak için kullanılan bir optimizasyon yöntemidir. Bu yöntem, hedef fonksiyonunun maksimum ya da minimum değerinin bulunmasını gerektiren matematiksel problemleri çözmek için kullanılır. Optimizasyon yöntemleri, belirli bir hedefi karşılamak için uygun parametreleri bulmak için iteratif olarak çalışırlar. Her adımda, parametreler güncellenir ve hedef fonksiyonu yeniden hesaplanır. Bu süreç, hedef fonksiyonunun minimum değerine ulaşana kadar devam eder.

 \textbf{Nöronların durumlarını optimize etmek ne anlama geliyor? }

 Nöronların durumlarını optimize etmek, bir nöral ağın belirli bir hedefe ulaşması için nöronların içsel durumlarının uygun bir şekilde ayarlanması anlamına gelir. Nöronların durumları, nöronların aktivasyon seviyeleri olarak ifade edilebilir.

Bir nöral ağın hedefi genellikle, girdi verilerinden çıktı verilerinin elde edilmesidir. Bu amaçla, nöronların durumları, ağın girdi verilerine yanıt olarak uygun şekilde ayarlanmalıdır. Bu, nöronların aktivasyon seviyelerinin, ağın girdi verileriyle uyumlu bir şekilde düzenlenmesini gerektirir.

Nöronların durumlarının optimize edilmesi, genellikle bir optimizasyon problemi olarak ele alınır. Bu problem, bir hedef fonksiyonunu (örneğin, ağın girdi verilerine yanıt olarak ürettiği çıktıların hata miktarını) minimize etmek için nöronların durumlarının uygun şekilde ayarlanmasını gerektirir. Bu işlem, genellikle bir eğitim süreci sırasında gerçekleştirilir ve nöronların durumları, bir dizi iteratif güncelleme adımıyla değiştirilir.

Nöronların durumlarının optimize edilmesi, yapay sinir ağlarının temel işleyiş prensiplerinden biridir ve ağların doğru şekilde eğitilmesi için önemlidir.



\subsection{(8 Puan)} \textbf{ChatGPT’nin açıklamalarından ne anladığınızı buraya kendi kelimelerinizle özetleyin. Örneğin ``Hopfield Network nedir? Ne işe yarar? Neden bilmemiz gerekir bunu? Gerçek hayatta kullanılıyor mu?'' gibi. Anlamadığınız kısımlar varsa ve ChatGPT’nin açıklamaları yeterli gelmezse internet dahil farklı kaynaklardan araştırıp, bilgilerin doğruluğunu sorgulamakta serbestsiniz. Konuyu doğru anlayıp anlamamanız puana etki edecektir. Kullandığınız kaynakları da belirtin!}


Hopfield ağları John Hopfield tarafından geliştirilmiş, çoğunlukla hatırlama ve optimizasyon problemleri için kullanılan yapay sinir ağı türüdür. Hopfield ağları konusu yapay sinir ağlarında önemli bir yere sahiptir bu yüzden yapay zeka alanında kariyer yapmak isteyenlerin bilmesi gerekir.
Hopfield ağları birbirine bağlı nöronlardan oluşur ve bir nöron diğer tüm nöronlarla bağlıdır.Bir nöronun çıktısı diğer nöronların girdisi olabilir. Bu sayede agın davranışı bir bütün olarak incelenir ve bir nöronun durumu diğer nöronların durumuna bağlı olarak değişebilir. Hatırlama işlemi sırasında bir nöronun durumu diğer nöronların durumlarına bağlı olarak değişir.
Kolay kullanımı sebebiyle küçük çaplı problemler için yaygın olarak kullanılır fakat büyük problemler için yeterli performansı sağlayamazlar. Günlük hayattan bazı kullanım örnekleri vermek gerekirse bir fabrikanın bölümlerindeki makinelerin çalışma zamanlarını optimize etmek için, bir insan yüzünü veya bir kelimeyi hatırlamak için hafıza belleklerinde, resim, video, ses gibi veri sıkıştırma algoritmalarında kullanılabilir.

Kaynak: chatgpt
\section{(Toplam 10 Puan) Boltzman Machine:}

\subsection{(2 Puan)} \textbf{ChatGPT’ye size Boltzman Machine ilgili detaylı bilgi vermesini isteyin. Sorduğunuz soruyu, ChatGPT'nin çıkartığı cevabı aşağı koyun. Bu cevapta anlamadığınız terimleri de ChatGPT’ye sorup, sorduğunuz soruları ve ChatGPT’nin verdiği cevapları da sırayla aşağı koyun.}

\textbf{Boltzman Machine hakkında detaylı bilgi ver.}

Boltzmann makinesi, yapay sinir ağlarının bir türüdür ve özellikle makine öğrenmesinde kullanılır. Adını fizikçi Ludwig Boltzmann'dan almıştır.

Bir Boltzmann makinesi, birbirleriyle bağlantılı bir dizi "gizli" veya "görünmez" nöron ve bunlardan oluşan bir "görülebilir" nöron seti ile karakterize edilir. Görülebilir nöronlar, ağın dış dünyadan aldığı girdileri temsil ederken, gizli nöronlar ağın iç yapısını temsil eder. Ağın her nöronu, diğer tüm nöronlara bağlanır ve bu bağlantılar ağı öğrenmeye ve girdilere yanıt vermeye yönlendiren ağırlıklarla belirlenir.

Bir Boltzmann makinesi, bir veri kümesinin özelliklerini öğrenmek için kullanılabilir. Ağın öğrenmesi, veri kümesindeki örneklerin ağırlıklarını ayarlayarak gerçekleştirilir. Öğrenme süreci, ağın verilen bir girdiye tepki verdiği ve ardından ağırlıkların ayarlandığı birkaç turdan oluşur. Bu, ağın, verilen girdilere göre daha iyi bir performans göstermek için kendisini ayarlamasına olanak tanır.

Bir Boltzmann makinesi, unsupervised learning (gözetimsiz öğrenme) için özellikle yararlıdır, yani veri kümesinin etiketleri yokken kullanılabilir. Ayrıca, belirli bir veri kümesine özgü olmayan özellikleri keşfetme ve benzerlikleri öğrenme yeteneğiyle öne çıkar.

Boltzmann makinesi, özellikle karmaşık veri kümelerinde ve yüksek boyutlu veri kümelerinde etkilidir. Ancak, öğrenme süreci uzun sürebilir ve büyük veri kümeleri için hesaplama gücü ve bellek gereksinimleri yüksek olabilir.

\textbf{Gözetimsiz öğrenme nedir?}

Gözetimsiz öğrenme, makine öğrenmesinin bir dalıdır ve özellikle etiketlenmemiş verilerin özelliklerini öğrenmek için kullanılır. Bu yöntem, önceden bilinen etiketli verilerin olmaması durumunda kullanılır.

Gözetimsiz öğrenme, veri kümesindeki yapıyı ve özellikleri anlamak için kullanılan bir yöntemdir. Bu yöntemde, makine, verilerin özelliklerini kendisi belirler ve verileri gruplara ayırır. Böylece, makine, verilerdeki belirli kalıpları ve yapıları keşfeder ve verileri işlemek için gerekli olan bilgiyi elde eder.

Gözetimsiz öğrenme, özellikle büyük veri kümelerinde, veri setleri hakkında daha fazla bilgi edinmek için kullanışlıdır. Bu yöntem, özellikle belirli bir veri kümesine özgü olmayan özellikleri keşfetme ve benzerlikleri öğrenme yeteneğiyle öne çıkar. Ayrıca, gözetimsiz öğrenme, denetimli öğrenme için bir öncül olabilir, yani öncelikle gözetimsiz öğrenme kullanarak veri kümesindeki yapıyı ve kalıpları keşfettikten sonra denetimli öğrenme yöntemleriyle daha yüksek doğruluk elde etmek mümkün olabilir.

Gözetimsiz öğrenme yöntemleri arasında kümeleme, boyut azaltma ve öznitelik öğrenme gibi teknikler bulunmaktadır.

\textbf{Denetimli öğrenme nedir?}

Denetimli öğrenme, makine öğrenmesinin bir dalıdır ve önceden etiketlenmiş verilerin kullanıldığı bir öğrenme yöntemidir. Bu yöntemde, makine, verilerdeki belirli kalıpları ve ilişkileri anlamak için kullanılır.

Denetimli öğrenme, verilerdeki belirli bir kalıbı veya ilişkiyi öğrenmek için kullanılır. Bu yöntemde, veri kümesi önceden etiketlenmiş verilerden oluşur, yani her veri örneği bir girdi ve bir çıktı etiketi içerir. Makine, bu veri kümesi üzerinde eğitilir ve girdilere dayanarak doğru çıktıları üretmeyi öğrenir.

Denetimli öğrenme, sınıflandırma ve regresyon problemleri için kullanılabilir. Sınıflandırma problemleri, verileri farklı sınıflara ayırmak için kullanılırken, regresyon problemleri, veriler arasındaki ilişkileri modeller.

Denetimli öğrenme yöntemleri arasında yapay sinir ağları, karar ağaçları, destek vektör makineleri, en yakın komşu algoritmaları ve doğrusal regresyon gibi teknikler bulunmaktadır.

Denetimli öğrenme yöntemi, eğitim verilerindeki etiketli verilere göre doğru tahminler yapabilen modeller oluşturabilir. Ancak, bu yöntemin dezavantajı, eğitim verileri ile sınırlı olmasıdır. Bu nedenle, veri kümesi genellemeleri açısından yeterli olmayabilir. Ayrıca, eğitim verilerinin kalitesi ve doğruluğu da sonuçların doğruluğunu etkileyebilir.

\subsection{(8 Puan)} \textbf{ChatGPT’nin açıklamalarından ne anladığınızı buraya kendi kelimelerinizle özetleyin. Örneğin ``Boltzman Machine nedir? Ne işe yarar? Neden bilmemiz gerekir bunu? Gerçek hayatta kullanılıyor mu?'' gibi. Anlamadığınız kısımlar varsa ve ChatGPT’nin açıklamaları yeterli gelmezse internet dahil farklı kaynaklardan araştırıp, bilgilerin doğruluğunu sorgulamakta serbestsiniz. Konuyu doğru anlayıp anlamamanız puana etki edecektir. Kullandığınız kaynakları da belirtin!}

Boltzman Makinesi Ludwig Boltzmann tarafından geliştirilen bir yapay sinir ağı türüdür ve makine öğrenmesinde kullanılır. Boltzman makinesi gizli ve görülebilir nöronlardan oluşur. Gizli nöronlar ağın iç yapısını, görülebilir nöronlar ağın dış dünyadan aldığı girdileri temsil eder. Burada da her bir nöron diğer nöronlarla bağlantılıdır ve bu bağlantılar bir ağırlık içerir. Boltzman makinesi önceden bilinen etiketli verilerin olmadığı gözetimsiz öğrenme durumlarında kullanılabilir. Karmaşık ve yüksek boyutlu veri kümelerinde etkilidir. Gündelik hayattan kullanım örnekleri: bir e-ticaret sitesinde kullanıcının önceki alışverişlerine bakarak ilgilenebileceği ürün önerileri sunma, bir metnin devamında ne olabileceği hakkında tahmin, görüntü üzerinden nesne tanıma gibi durumlarda kullanılabilir.

Kaynak: chatgpt

\section{(Toplam 10 Puan) Markov Assumption ve Markov Chain:}

\subsection{(2 Puan)} \textbf{ChatGPT’ye size Markov assumption ve Markov Chain ilgili detaylı bilgi vermesini isteyin. Sorduğunuz soruyu, ChatGPT'nin çıkartığı cevabı aşağı koyun. Bu cevapta anlamadığınız terimleri de ChatGPT’ye sorup, sorduğunuz soruları ve ChatGPT’nin verdiği cevapları da sırayla aşağı koyun.}

\textbf{ Markov Assumption nedir?}

Markov varsayımı, bir sürecin gelecekteki durumlarının, yalnızca mevcut durumundan bağımsız olduğu varsayımıdır. Bu varsayıma göre, bir sürecin gelecekteki durumları, mevcut durumunun bilgisi ile tamamen belirlenebilir.

Markov varsayımı, özellikle Markov zincirleri olarak bilinen süreçlerde kullanılır. Markov zincirleri, bir dizi durumdan oluşan bir süreçtir ve her bir durum, yalnızca bir sonraki durumun belirlenmesinde önemlidir.

Markov varsayımı, birçok uygulama alanında kullanılır. Örneğin, doğal dil işlemede, bir kelimenin gelecekteki olası kullanımları, yalnızca mevcut kelimenin bağlamına bağlıdır. Benzer şekilde, finansal piyasalarda, gelecekteki fiyatlar yalnızca mevcut fiyatların bilgisine dayalıdır.

Markov varsayımı, bir dizi matematiksel ve istatistiksel yöntemle birlikte kullanılır. Bu yöntemler, Markov zincirlerinin özelliklerini ve davranışlarını analiz etmek için kullanılır ve gelecekteki durumların olasılıklarını tahmin etmek için kullanılabilir.

Ancak, Markov varsayımı gerçek dünya problemlerinde her zaman geçerli değildir. Gerçek dünyadaki birçok süreç, geçmiş durumlardan etkilenir ve gelecekteki durumlar yalnızca mevcut durumdan bağımsız değildir. Bu nedenle, Markov varsayımı, uygulamalarında dikkatli bir şekilde kullanılmalı ve sınırlandırmaları anlaşılmalıdır.

\textbf{Markov Chain nedir?}

Markov zinciri, bir dizi durumdan oluşan bir süreçtir ve her bir durumun oluşması, yalnızca bir önceki duruma bağlıdır. Yani, bir durumun olasılığı, sadece bir önceki durumun bilgisine dayalıdır ve geçmiş durumlar ile ilgili bilgi gerektirmez. Bu özellik, Markov zincirlerine "Markov özelliği" veya "Markov varsayımı" adını verir.

Markov zincirleri, çeşitli uygulama alanlarında kullanılır. Örneğin, doğal dil işlemede, bir kelimenin olası kullanımları, yalnızca mevcut kelimenin bağlamına bağlıdır ve bu, bir Markov zinciri olarak modellenebilir. Finansal piyasalarda, fiyat hareketleri de Markov zincirleri olarak modellenebilir.

Markov zincirleri, bir dizi matematiksel yöntemle birlikte kullanılır. Bu yöntemler, Markov zincirlerinin özelliklerini ve davranışlarını analiz etmek için kullanılır ve gelecekteki durumların olasılıklarını tahmin etmek için kullanılabilir.

Markov zincirleri aynı zamanda makine öğrenmesi ve yapay zeka alanlarında da yaygın olarak kullanılır. Örneğin, doğal dil işlemede, dil modelleri oluşturmak için Markov zincirleri kullanılır. Ayrıca, Markov zincirleri, sınıflandırma, kümeleme ve veri boyutu azaltma gibi problemleri çözmek için de kullanılabilir.

Markov zincirleri, Markov özelliğine sahip olan birçok doğal olayın modellenmesi için uygun bir seçenek olabilir. Ancak, gerçek dünya problemlerinde Markov özelliği her zaman geçerli değildir ve bu nedenle Markov zincirleri, uygulamalarında dikkatli bir şekilde kullanılmalıdır.

\subsection{(8 Puan)} \textbf{ChatGPT’nin açıklamalarından ne anladığınızı buraya kendi kelimelerinizle özetleyin. Örneğin ``Markov assumption ve Markov Chain nedir? Ne işe yarar? Neden bilmemiz gerekir bunu? Gerçek hayatta kullanılıyor mu?'' gibi. Anlamadığınız kısımlar varsa ve ChatGPT’nin açıklamaları yeterli gelmezse internet dahil farklı kaynaklardan araştırıp, bilgilerin doğruluğunu sorgulamakta serbestsiniz. Konuyu doğru anlayıp anlamamanız puana etki edecektir. Kullandığınız kaynakları da belirtin!}

Markov varsayımı ileride ortaya çıkması muhtemel durumların geçmiş verilerden değil şuanki verilerden yararlanılarak bulunmasını varsayar. Markov varsayımı Markov Zincirleri adı verilen süreçlerde kullanılır. Bu süreçlerdeki her bir durum bir sonraki durumun belirlenmesinde önemlidir. Yani sürecin bir durumdan diğer duruma geçişi daha önceki durumlara değil yalnızca bir önceki duruma bağlıdır. Bu yüzden bir önceki durum hariç diğer durumların önemi yoktur.
Günlük hayattan kullanım örnekleri: dava durumu tahminleri, finansal analizler, yolculuk planlama gibi alanlarda kullanılabilir. 

Kaynaklar: chatgpt, wikipedia, https://dergipark.org.tr/en/download/article-file/17815

\section{(Toplam 20 Puan) Feed Forward:}
 
\begin{itemize}
    \item Forward propagation için, input olarak şu X matrisini verin (tensöre çevirmeyi unutmayın):\\
    $X = \begin{bmatrix}
        1 & 2 & 3\\
        4 & 5 & 6
        \end{bmatrix}$
    Satırlar veriler (sample'lar), kolonlar öznitelikler (feature'lar).
    \item Bir adet hidden layer olsun ve içinde tanh aktivasyon fonksiyonu olsun
    \item Hidden layer'da 50 nöron olsun
    \item Bir adet output layer olsun, tek nöronu olsun ve içinde sigmoid aktivasyon fonksiyonu olsun
\end{itemize}

Tanh fonksiyonu:\\
$f(x) = \frac{exp(x) - exp(-x)}{exp(x) + exp(-x)}$
\vspace{.2in}

Sigmoid fonksiyonu:\\
$f(x) = \frac{1}{1 + exp(-x)}$

\vspace{.2in}
 \textbf{Pytorch kütüphanesi ile, ama kütüphanenin hazır aktivasyon fonksiyonlarını kullanmadan, formülünü verdiğim iki aktivasyon fonksiyonunun kodunu ikinci haftada yaptığımız gibi kendiniz yazarak bu yapay sinir ağını oluşturun ve aşağıdaki üç soruya cevap verin.}
 
\subsection{(10 Puan)} \textbf{Yukarıdaki yapay sinir ağını çalıştırmadan önce pytorch için Seed değerini 1 olarak set edin, kodu aşağıdaki kod bloğuna ve altına da sonucu yapıştırın:}

% Latex'de kod koyabilirsiniz python formatında. Aşağıdaki örnekleri silip içine kendi kodunuzu koyun
\begin{python}
kod_buraya = None
if kod_buraya:
    devam_ise_buraya = 0

print(devam_ise_buraya)
\end{python}

Sonuç da buraya

\subsection{(5 Puan)} \textbf{Yukarıdaki yapay sinir ağını çalıştırmadan önce Seed değerini öğrenci numaranız olarak değiştirip, kodu aşağıdaki kod bloğuna ve altına da sonucu yapıştırın:}

\begin{python}
kod_buraya = None
if kod_buraya:
    devam_ise_buraya = 0

print(devam_ise_buraya)
\end{python}

Sonuç da buraya

\subsection{(5 Puan)} \textbf{Kodlarınızın ve sonuçlarınızın olduğu jupyter notebook'un Github repository'sindeki linkini aşağıdaki url kısmının içine yapıştırın. İlk sayfada belirttiğim gün ve saate kadar halka açık (public) olmasın:}
% size ait Github olmak zorunda, bu vize için ayrı bir github repository'si açıp notebook'u onun içine koyun. Kendine ait olmayıp da arkadaşının notebook'unun linkini paylaşanlar 0 alacak.

\url{www.githublinkiburaya.com}

\section{(Toplam 40 Puan) Multilayer Perceptron (MLP):} 
\textbf{Bu bölümdeki sorularda benim vize ile beraber paylaştığım Prensesi İyileştir (Cure The Princess) Veri Seti parçaları kullanılacak. Hikaye şöyle (soruyu çözmek için hikaye kısmını okumak zorunda değilsiniz):} 

``Bir zamanlar, çok uzaklarda bir ülkede, ağır bir hastalığa yakalanmış bir prenses yaşarmış. Ülkenin kralı ve kraliçesi onu iyileştirmek için ellerinden gelen her şeyi yapmışlar, ancak denedikleri hiçbir çare işe yaramamış.

Yerel bir grup köylü, herhangi bir hastalığı iyileştirmek için gücü olduğu söylenen bir dizi sihirli malzemeden bahsederek kral ve kraliçeye yaklaşmış. Ancak, köylüler kral ile kraliçeyi, bu malzemelerin etkilerinin patlayıcı olabileceği ve son zamanlarda yaşanan kuraklıklar nedeniyle bu malzemelerden sadece birkaçının herhangi bir zamanda bulunabileceği konusunda uyarmışlar. Ayrıca, sadece deneyimli bir simyacı bu özelliklere sahip patlayıcı ve az bulunan malzemelerin belirli bir kombinasyonunun prensesi iyileştireceğini belirleyebilecekmiş.

Kral ve kraliçe kızlarını kurtarmak için umutsuzlar, bu yüzden ülkedeki en iyi simyacıyı bulmak için yola çıkmışlar. Dağları tepeleri aşmışlar ve nihayet "Yapay Sinir Ağları Uzmanı" olarak bilinen yeni bir sihirli sanatın ustası olarak ün yapmış bir simyacı bulmuşlar.

Simyacı önce köylülerin iddialarını ve her bir malzemenin alınan miktarlarını, ayrıca iyileşmeye yol açıp açmadığını incelemiş. Simyacı biliyormuş ki bu prensesi iyileştirmek için tek bir şansı varmış ve bunu doğru yapmak zorundaymış. (Original source: \url{https://www.kaggle.com/datasets/unmoved/cure-the-princess})

(Buradan itibaren ChatGPT ve Dr. Ulya Bayram'a ait hikayenin devamı)

Simyacı, büyülü bileşenlerin farklı kombinasyonlarını analiz etmek ve denemek için günler harcamış. Sonunda birkaç denemenin ardından prensesi iyileştirecek çeşitli karışım kombinasyonları bulmuş ve bunları bir veri setinde toplamış. Daha sonra bu veri setini eğitim, validasyon ve test setleri olarak üç parçaya ayırmış ve bunun üzerinde bir yapay sinir ağı eğiterek kendi yöntemi ile prensesi iyileştirme ihtimalini hesaplamış ve ikna olunca kral ve kraliçeye haber vermiş. Heyecanlı ve umutlu olan kral ve kraliçe, simyacının prensese hazırladığı ilacı vermesine izin vermiş ve ilaç işe yaramış ve prenses hastalığından kurtulmuş.

Kral ve kraliçe, kızlarının hayatını kurtardığı için simyacıya krallıkta kalması ve çalışmalarına devam etmesi için büyük bir araştırma bütçesi ve çok sayıda GPU'su olan bir server vermiş. İyileşen prenses de kendisini iyileştiren yöntemleri öğrenmeye merak salıp, krallıktaki üniversitenin bilgisayar mühendisliği bölümüne girmiş ve mezun olur olmaz da simyacının yanında, onun araştırma grubunda çalışmaya başlamış. Uzun yıllar birlikte krallıktaki insanlara, hayvanlara ve doğaya faydalı olacak yazılımlar geliştirmişler, ve simyacı emekli olduğunda prenses hem araştırma grubunun hem de krallığın lideri olarak hayatına devam etmiş.

Prenses, kendisini iyileştiren veri setini de, gelecekte onların izinden gidecek bilgisayar mühendisi prensler ve prensesler başkalarına faydalı olabilecek yapay sinir ağları oluşturmayı öğrensinler diye halka açmış ve sınavlarda kullanılmasını salık vermiş.''

\textbf{İki hidden layer'lı bir Multilayer Perceptron (MLP) oluşturun beşinci ve altıncı haftalarda yaptığımız gibi. Hazır aktivasyon fonksiyonlarını kullanmak serbest. İlk hidden layer'da 100, ikinci hidden layer'da 50 nöron olsun. Hidden layer'larda ReLU, output layer'da sigmoid aktivasyonu olsun.}

\textbf{Output layer'da kaç nöron olacağını veri setinden bakıp bulacaksınız. Elbette bu veriye uygun Cross Entropy loss yöntemini uygulayacaksınız. Optimizasyon için Stochastic Gradient Descent yeterli. Epoch sayınızı ve learning rate'i validasyon seti üzerinde denemeler yaparak (loss'lara overfit var mı diye bakarak) kendiniz belirleyeceksiniz. Batch size'ı 16 seçebilirsiniz.}

\subsection{(10 Puan)} \textbf{Bu MLP'nin pytorch ile yazılmış class'ının kodunu aşağı kod bloğuna yapıştırın:}

\begin{python}
kod_buraya = None
if kod_buraya:
    devam_ise_buraya = 0

print(devam_ise_buraya)
\end{python}

\subsection{(10 Puan)} \textbf{SEED=öğrenci numaranız set ettikten sonra altıncı haftada yazdığımız gibi training batch'lerinden eğitim loss'ları, validation batch'lerinden validasyon loss değerlerini hesaplayan kodu aşağıdaki kod bloğuna yapıştırın ve çıkan figürü de alta ekleyin.}

\begin{python}
kod_buraya = None
if kod_buraya:
    devam_ise_buraya = 0

print(devam_ise_buraya)
\end{python}

% Figure aşağıda comment içindeki kısımdaki gibi eklenir.
\begin{comment}
\begin{figure}[ht!]
    \centering
    \includegraphics[width=0.75\textwidth]{mypicturehere.png}
    \caption{Buraya açıklama yazın}
    \label{fig:my_pic}
\end{figure}
\end{comment}

\subsection{(10 Puan)} \textbf{SEED=öğrenci numaranız set ettikten sonra altıncı haftada ödev olarak verdiğim gibi earlystopping'deki en iyi modeli kullanarak, Prensesi İyileştir test setinden accuracy, F1, precision ve recall değerlerini hesaplayan kodu yazın ve sonucu da aşağı yapıştırın. \%80'den fazla başarı bekliyorum test setinden. Daha düşükse başarı oranınız, nerede hata yaptığınızı bulmaya çalışın. \%90'dan fazla başarı almak mümkün (ben denedim).}

\begin{python}
kod_buraya = None
if kod_buraya:
    devam_ise_buraya = 0

print(devam_ise_buraya)
\end{python}

Sonuçlar buraya

\subsection{(5 Puan)} \textbf{Tüm kodların CPU'da çalışması ne kadar sürüyor hesaplayın. Sonra to device yöntemini kullanarak modeli ve verileri GPU'ya atıp kodu bir de böyle çalıştırın ve ne kadar sürdüğünü hesaplayın. Süreleri aşağıdaki tabloya koyun. GPU için Google Colab ya da Kaggle'ı kullanabilirsiniz, iki ortam da her hafta saatlerce GPU hakkı veriyor.}

\begin{table}[ht!]
    \centering
    \caption{Buraya bir açıklama yazın}
    \begin{tabular}{c|c}
        Ortam & Süre (saniye) \\\hline
        CPU & kaç? \\
        GPU & kaç?\\
    \end{tabular}
    \label{tab:my_table}
\end{table}

\subsection{(3 Puan)} \textbf{Modelin eğitim setine overfit etmesi için elinizden geldiği kadar kodu gereken şekilde değiştirin, validasyon loss'unun açıkça yükselmeye başladığı, training ve validation loss'ları içeren figürü aşağı koyun ve overfit için yaptığınız değişiklikleri aşağı yazın. Overfit, tam bir çanak gibi olmalı ve yükselmeli. Ona göre parametrelerle oynayın.}

Cevaplar buraya

% Figür aşağı
\begin{comment}
\begin{figure}[ht!]
    \centering
    \includegraphics[width=0.75\textwidth]{mypicturehere.png}
    \caption{Buraya açıklama yazın}
    \label{fig:my_pic}
\end{figure}
\end{comment}

\subsection{(2 Puan)} \textbf{Beşinci soruya ait tüm kodların ve cevapların olduğu jupyter notebook'un Github linkini aşağıdaki url'e koyun.}

\url{www.benimgithublinkim.com}

\section{(Toplam 10 Puan)} \textbf{Bir önceki sorudaki Prensesi İyileştir problemindeki yapay sinir ağınıza seçtiğiniz herhangi iki farklı regülarizasyon yöntemi ekleyin ve aşağıdaki soruları cevaplayın.} 

\subsection{(2 puan)} \textbf{Kodlarda regülarizasyon eklediğiniz kısımları aşağı koyun:} 

\begin{python}
kod_buraya = None
if kod_buraya:
    devam_ise_buraya = 0

print(devam_ise_buraya)
\end{python}

\subsection{(2 puan)} \textbf{Test setinden yeni accuracy, F1, precision ve recall değerlerini hesaplayıp aşağı koyun:}

Sonuçlar buraya.

\subsection{(5 puan)} \textbf{Regülarizasyon yöntemi seçimlerinizin sebeplerini ve sonuçlara etkisini yorumlayın:}

Yorumlar buraya.

\subsection{(1 puan)} \textbf{Sonucun github linkini  aşağıya koyun:}

\url{www.benimgithublinkim2.com}

\end{document}
